\begin{quote}
    \itshape
    \setlength{\leftskip}{1cm}
    By providing long-run guidance, the central bank may influence long-term interest rates. \\
    \normalfont
    \hspace*{\fill}--- Isabel Schnabel, Member of the Executive Board of the ECB
\end{quote}

\section{Introduction}

The declining trend of interest rates in the past few decades is striking. Although some scholars discussed that the slump in the real interest rates could be traced back for centuries \citep{rogoff2022}, there has been, at least, more or less a consensus on the secular stagnation of the last few decades since the IMF Conference in 2013. Economists suggested a variety of explanations for the secular decline, including demand-side explanations such as demand shortfall along with aging population \citep{krugman2014four}, ``a lack of investment opportunities'' \citep{summers2014reflections}, as well as supply-side explanations such as the slump in productivity growth \citep{gordon2017rise}. While these prominent explanations account for the economic forces that are beyond the reach of monetary policy, in a recent and striking study, \citet{hillenbrand2022} found that 3-day time windows around the Federal Open Market Committee (FOMC) meetings capture the secular decline in 10-year Treasuries in the past few decades, and outside-window yield changes are transitory. By courtesy of this remarkable result, the question naturally arises is the following: In the world of U.S. dollar dominated debt markets, do the monetary policy decisions of central banks other than the Fed have similar explanatory power on the yield change in long-term government bonds, or is Hillenbrand's (2022) result a unique case for the Fed? By extension, can the observed secular decline in the real interest rates across other countries, depicted in the yield change of the long-term government bonds, be attributed to the monetary policy decisions of the Fed, thereby leading to a potential discourse on the Global Financial Cycle? \\

The results of this paper indicate that although there is a strong heterogeneity between advanced economies, due to the economic variables such as unconventional monetary policy tools, exchange rate interventions and financial frictions, to a certain extent, there is a limited but supporting evidence for a stronger Fed effect than the decisions of other countries' central banks around the narrow time windows over other countries' government bond yields. In this respect, this evidence is in line with \citet{miranda2020us} that indicating single global factor
explains around one-fifth of a common variation of the risky asset prices around the world. \\

In a nutshell, in this paper, following the descriptive yet stimulating approach of \citet{hillenbrand2022}, I constructed 3-day monetary policy decision windows around the 10-year government bond yields for selected advanced economies, using both the decision dates of the national central bank and the Fed. This approach allowed me to capture yield movements around both central banks' decisions, thus proposing evidence to confirm or reject the Global Financial Cycle hypothesis. Then, the simple empirical approach developed in this study is an OLS method to model the spillovers of bond yields from the 10-year Treasuries to selected countries' 10-year bonds around the 3-day Fed windows. Lastly, financial frictions are incorporated into the model in order to eliminate the omitted variable bias. With an attempt to establish a linkage between the Global Financial Cycle and the secular decline of interest rates, this paper departs from similar studies.