\section{Related Literature}

The secular decline is noted by Summers paper belki buraya. Moreover, \citet{rogoff2022} claimed that the decline in interest rates can be traced back until 1300s, reflecting the reduced discount/risk factors globally over time. \\
% emin değilim bu mu, paperı oku, götünden sallama

In their study, \citet{gilchrist2014us} presented that the longer-term interest rates in advanced economies, proxied by 10-year bond yields, were declined in response to both an unanticipated conventional easing and unconventional monetary actions of the Fed. While conventional monetary easing steepens the yield curve in advanced economies through a larger decline in the short-end of the yield curve, unconventional monetary actions narrows the yield spread of nominal foreign interest rates down. \\

% hillenbrand'ı kopyala yapıştır yaptım
\citet{hillenbrand2022} states that a narrow window around Fed meetings captures the secular decline in U.S. Treasury yields since 1980. Yield movements outside this window are transitory and wash out over time. This is surprising because the forces behind the secular decline are thought to be independent of monetary policy. However, Fed announcements might provide guidance about the long-run path of interest rates. In direct support of such“Long-run Fed Guidance”. \\

\citet{hanson2015monetary} documented that the changes in monetary policy affect the 10-year forward real rates, utilizing movements during the FOMC announcement days. They offer a ``reaching for yield'' mechanism such that the yield-oriented investors substitute to longer-term bonds as short-term yields decline, if the yield curve is upward-sloping. In turn, increasing demand for the longer maturity bonds leads to increasing prices and declining yields. That is, this explanation relies on a ``term premium'' effect. \citep{hanson2015monetary}\\

Another line of research, namely on the ``Global Financial Cycle'', is reflected in \citet{miranda2020us} and \citet{rey2021}. \citet{miranda2020us} demonstrated that  a single global factor explains around one-fifth of a common variation of the risky asset prices around the world. Given that the U.S. dollar is the dominant currency of global banking, one instance of this that the almost \%x0 of the syndicated loans between x and x is denominated in the U.S. dollar, the monetary policy decisions by the Fed has a direct impact over the Global Financial Cycle. The potential explanations for this phenomenon is the deleveraging of the financial intermediaries around the globe, and relatedly, decline in global credit and gross capital flows, and a significant rise in aggregate risk aversion. \citep{miranda2020us} \\

In their work on surrender options in life insurance and market interest rates, \citet{kubitza2023life} estimates two-stage least-squares regressing German government bond rates on the U.S. monetary policy shocks, claiming a transmission through the international bond market channel.
