\section{Data and Institutional Background}

This section involves providing background information on the institutional background of the central banks' mechanisms for monetary decision-making for each country in the sample. Later, I elaborate on both yield data and monetary policy decision data, including the number of observations and time intervals.

\subsection{Eurozone}

In the European Central Bank (ECB), the Governing Council is the principal decision-making entity for conducting monetary policy. The Governing Council consists of twenty-six members---six members of the Executive Board and the Euro-area national central bank governors. While the Governing Council members meet twice a month to evaluate macroeconomic and financial conditions, it decides monetary policy stance every six weeks. The Governing Council conducts monetary policy through three key interest rates: the main refinancing operations rate, the deposit facility rate, and the marginal lending facility rate. I obtained the dates of monetary policy decisions from the ECB website. My sample contains in total 299 monetary policy decision dates, from March 1999 to March 2024. To remove potential heterogeneity between bond yields of different countries, I selected Germany. I collected on-the-run yield data on 10-year German Bunds.

\subsection{United Kingdom}

In the United Kingdom, the Monetary Policy Committee (MPC) is the key decision-making body of the Bank of England (BoE) to conduct monetary policy. The MPC is made up of nine members – the Governor, the three Deputy Governors for Monetary Policy, Financial Stability and Markets and Banking, our Chief Economist and four external members appointed directly by the Chancellor. The main monetary policy interest rate set by the MPC is called the `Bank Rate', which refers to the interest rate BoE pays to commercial banks that hold money with the BoE. The dates of monetary policy decisions are collected from the BoE's voting history database, ranging from June 1997 to March 2024, and the total number of data points is 295. I collected yield data of the UK government bonds, also known as gilts, from 1979 to 2024 from the BoE database. The yield data contains the estimation for zero-coupon continuously-compounded yields, computations are elaborated in \citet{anderson2001new}.

\subsection{Japan}
In Japan, the key decision-making body for conducting monetary policy is the Policy Board of the Bank of Japan (BoJ). The Policy Board consists of nine members, including the Governor, two Deputy Governors, and six other members who are appointed by the Cabinet. The primary monetary policy instrument used by the BoJ is the Policy-Rate Balance, which refers to the interest rate applied to the policy-rate balances held by financial institutions at the BoJ. In the dataset on monetary policy decision dates by the BoJ, there are 515 data points, ranging from December 1981 to March 2024. The interest rates on government bonds are collected from the database of Japanese Ministry of Finance, and computed using on-the-run securities.


\subsection{Canada}

In Canada, the Governing Council of the Bank of Canada is the decision-making body for conducting monetary policy and promoting a more resilient financial system. The Governing Council comprises six members: the Governor, the Senior Deputy Governor, and four Deputy Governors. The Council uses the policy interest rate as a main tool for conducting monetary policy, and the rate is typically set on eight predetermined announcement dates annually, with decisions reached by consensus rather than through individual votes. Within the sample, there are 175 monetary policy decision dates spanning from January 1999 to April 2024, and for the yield data, I consulted the Bank of Canada's own data sources, from which \citet{bolder2004empirical} constructed the historical zero-coupon yield curve data.


\subsection{Switzerland}
The Governing Board, the Swiss National Bank's (SNB) highest management and executive body, comprises three members appointed for six-year terms by the Federal Council upon the recommendation of the Bank Council. The Governing Board holds the primary responsibility for formulating monetary policy. The SNB implements this policy by setting the SNB policy rate, with the objective of aligning short-term Swiss franc money market rates closely with the policy rate. The data on the monetary policy decision dates spans from January 2000 to March 2024, with a total number of 109 meetings. Yield data, referred to as the spot interest rates, represent the yields on zero-coupon bonds computed by the extended Nelson-Siegel procedure by the SNB.

\subsection{Australia}

The Reserve Bank of Australia (RBA) conducts monetary policy through its key executive body, the Reserve Bank Board. This Board comprises nine members, including the Governor, the Deputy Governor, the Secretary to the Treasury, and six other members appointed by the government. The Board convenes eight times a year to review and set monetary policy. The primary monetary policy instrument utilized by the RBA is the ``cash rate target'', which is the interest rate on overnight loans in the money market, and this rate influences a range of interest rates across the economy. For my analysis, I collected the dates of monetary policy decisions from the RBA, spanning from August 1992 to March 2024, which includes a total of 215 meetings. Additionally, I collected yield data of Australian government bonds from January 1995 to May 2024 from the Reserve Bank of Australia's database. The yields are computed and interpolated by the RBA.
