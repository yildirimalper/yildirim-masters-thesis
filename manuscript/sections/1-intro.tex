\begin{quote}
    \itshape
    \setlength{\leftskip}{1cm}
    By providing long-run guidance, the central bank may influence long-term interest rates. \\
    \normalfont
    \hspace*{\fill}--- Isabel Schnabel, Member of the Executive Board of the ECB
\end{quote}

\section{Introduction}

The declining trend of interest rates in the past few decades is striking. Although some scholars discussed that the slump in the real interest rates could be traced back for centuries \citep{rogoff2022}, there has been, at least, more or less a consensus on the secular stagnation of the last few decades since the IMF Conference in 2013. Economists suggested a variety of explanations for the secular decline, including demand-side explanations such as demand shortfall along with aging population \citep{krugman2014four}, ``a lack of investment opportunities'' \citep{summers2014reflections}, as well as supply-side explanations such as the slump in productivity growth \citep{gordon2017rise}. While these prominent explanations account for the economic forces that are beyond the reach of monetary policy, in a recent and striking study, \citet{hillenbrand2022} found that 3-day time windows around the Federal Open Market Committee (FOMC) meetings capture the secular decline in 10-year Treasuries in the past few decades, and outside-window yield changes are transitory. By courtesy of this remarkable result, the question naturally arises is the following: In the world of US dollar dominated debt markets, do the monetary policy decisions of central banks other than the Fed have similar explanatory power on the yield change in long-term government bonds, or is Hillenbrand's (2022) result a unique case for the Fed? By extension, can the secular decline in the real interest rates in other countries, depicted in the yield change of the long-term government bonds, be accounted for by the monetary policy decisions of the Fed, leading to a potential discussion on a global monetary policy and the ``Global Financial Cycle''? \\

The results of my paper indicate that although there is a strong heterogeneity between advanced economies and confounders such as unconventional monetary policy tools or exchange rate interventions, to a certain extent, there is a supporting evidence for a stronger Fed effect than the national monetary policy over other countries' government bond yields. In this respect, this evidence is in line with \citet{miranda2015world} work on the World Asset Market bla bla. \\

In a nutshell, in this paper, following the descriptive yet stimulating approach of \citet{hillenbrand2022}, I constructed 3-day monetary policy decision windows around the 10-year government bond yields for selected advanced economies, using both the decision dates of the domestic central bank and the Fed. This approach allowed me to capture yield movements around both central banks that confirm or reject the Global Financial Cycle thesis. Then, the simple empirical approach developed in this study is an OLS method to confirm or deny at this covariance with certain statistical significance. With an attempt to establish a link between the literature on Global Financial Cycle and Secular Stagnation, this paper departs from the similar studies.