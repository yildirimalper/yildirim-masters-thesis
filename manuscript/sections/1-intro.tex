\begin{quote}
    \itshape
    \setlength{\leftskip}{1cm}
    By providing long-run guidance, the central bank may influence long-term interest rates. \\
    \normalfont
    \hspace*{\fill}--- Isabel Schnabel, Member of the Executive Board of the ECB
\end{quote}

\section{Introduction}

The declining trend of the interest rates in the past 50 years is striking. Although some authors discussed that the secular decline can be traced back for centuries. \citep{rogoff2022} In a recent study, \citet{hillenbrand2022} found that a 3-day time windows around the Federal Open Market Committee (FOMC) meetings capture the secular decline in 10-year Treasuries in the past few decades, and outside-window yield changes are transitory. In the face of this research, the natural question that arises is the following: Can the monetary policy decisions of central banks other than the Fed explain the change in long-term national bond yields, by providing a kind of ``long-term forward guidance'', or is this a unique case for the Fed? Furthermore, in relation to ``Global Financial Cycle" research \citep{miranda2020us, rey2021}, can the secular decline in other national long-term bond yields be accounted by the Fed's monetary policy decisions?