\section{Conclusion}

The findings of this paper contribute to the research field on the monetary policy announcements, interest rate dynamics, and the global financial cycle. The limited yet striking evidence suggests remarkable but heterogeneous influence of the Fed announcements over the long-term government bond yields of selected advanced economies. In Germany and Canada, for instance, there were significant cross-border monetary spillovers from the Fed announcements to long-term yields. Whereas, in Switzerland and Australia, the Fed effect over the long-term interest rates were substantially restricted. Throughout this paper, several potential explanatory variables to explain the sensitivity of national long-term bond yields to the Fed announcements have been discussed, namely, exchange rate interventions, unconventional monetary policy, and frictions in the FX markets. Nevertheless, except for the FX frictions, data unavailability prevented the research inquiry to explain the causes of varying degrees of spillover effects from being explored in more detail. \\

Although lacking statistical significance, the instrumental variable regression suggests that high transaction costs and low market liquidity have a restraining effect on the spillovers from Fed announcements to the long-term bonds of selected advanced economies. This means that the effect of the financial frictions in the global monetary spillovers could be inquired further. Moreover, a future research agenda could focus on a deeper understanding of the mechanisms underlying heterogeneous responses. For this purpose, a more detailed and extensive dataset of granular financial variables or high-frequency monetary policy shocks should be built. This would allow us to better understand how unconventional policies such as quantitative easing (QE) and exchange rate interventions have an impact on the spillovers from Fed announcements to the interest rates in the government bond markets. This would not only allow us to better understand the dynamics of global long-term real interest rates, but also make a valuable contribution to the discourse on the Global Financial Cycle.