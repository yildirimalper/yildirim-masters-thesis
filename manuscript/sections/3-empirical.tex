\section{Data and Institutional Background}

\subsection{Eurozone}

The Governing Council is the principal decision-making entity of the European Central Bank (ECB) for conducting monetary policy. The Governing Council consists of twenty-six members---six members of the Executive Board and the Euro-area national central bank governors. While the Governing Council members meet twice a month in order to evaluate macroeconomic and financial conditions, it decides monetary policy stance in every six weeks. The Governing Council conducts monetary policy through three key interest rates: the main refinancing operations rate, the deposit facility rate and the marginal lending facility rate. I obtained the dates of monetary policy decisions from the ECB website. My sample contains in total 299 monetary policy decision dates, from March 1999 to March 2024. \\

I collected yield data on European bonds, which includes \textit{nominal} interest rates of the AAA-rated government bonds, using the ECB Data Portal. The term structure data is modeled with the Svensson model, i.e., this data offers zero-coupon continuously-compounded yield curve.

\subsection{United Kingdom}

The Monetary Policy Committee (MPC) is the key decision-making body of the Bank of England (BoE) to conduct monetary policy. The MPC is made up of nine members – the Governor, the three Deputy Governors for Monetary Policy, Financial Stability and Markets and Banking, our Chief Economist and four external members appointed directly by the Chancellor. The main monetary policy interest rate set by the MPC is called `Bank Rate', which refers to the interest rate BoE pay to commercial banks that hold money with the BoE. I collected yield data of UK government bonds, also known as gilts, from XXXX to XXXX from the BoE database. Similarly, the yield data contains estimation for zero-coupon continuously-compounded yields. \citet{anderson2001new}

\subsection{Switzerland}
% copy paste hepsi https://www.snb.ch/en/the-snb/mandates-goals/monetary-policy
The Governing Board is the SNB's highest management and executive body. Its three members are appointed for a six-year term by the Federal Council on the recommendation of the Bank Council. The Governing Board is responsible, in particular, for monetary policy. The Swiss National Bank implements its monetary policy by setting the SNB policy rate. In so doing, it seeks to keep the short-term Swiss franc money market rates close to the SNB policy rate. These yields are known as spot interest rates, i.e. yields on zero-coupon bonds. Spot interest rates and/or the maturity/interest rate structure are estimated using the extended Nelson/Siegel procedure.

\subsection{Japan}
% copy paste https://www.boj.or.jp/en/mopo/outline/index.htm
The basic stance for monetary policy is decided by the Policy Board at Monetary Policy Meetings (MPMs). At MPMs, the Policy Board discusses the economic and financial situation, decides the guideline for money market operations and the Bank's monetary policy stance for the immediate future, and announces decisions immediately after the meeting concerned.

\subsection{Australia}

The Reserve Bank Board is the key decision-making body of the Reserve Bank of Australia. The Reserve Bank Board consists of nine members, of which at least five (including the Governor as Chair or Deputy Governor as Deputy Chair) must be present to conduct a meeting. The Board meets eight times a year. The Reserve Bank Board operates monetary policy through the `cash rate target', i.e., the interest rate on overnight loans. In sample, dates for the monetary policy decisions are ranged from August 1992 to March 2024, and in total, there are 215 meetings. Similarly, I collected yield data of Australian government bonds from July 1992 to May 2013 from the Reserve Bank of Australia's database. The data contains estimation for zero-coupon continuously-compounded yields.

\subsection{Canada}

\citet{bolder2004empirical} constructed the historical zero-coupon yield curve data.