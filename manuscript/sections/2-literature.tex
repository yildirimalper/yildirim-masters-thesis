\section{Related Literature}

The decline in real interest rates in recent decades is noted by \citet{summers2014reflections}, claiming that the lack of investment opportunities along with raised private saving propensities and reduced investment propensities. That is to say, this approach determines a macroeconomic framework around the Global Financial Crisis and the Eurozone crisis. On the other hand, \citet{rogoff2022} argues that the decline in the long-term real interest rates is not particular to recent decades. Instead, it is a trend stationary and persistent decline that dates back to the 1300s. This long-term decline, according to \citet{rogoff2022} reflects the reduced discount factors over time at the global scale and challenges explanations around productivity and demographics. \\

Shifting the perspective on the decline of real interest rates to financial markets and monetary policy, in their study, \citet{gilchrist2014us} presented that the longer-term interest rates in advanced economies, proxied by 10-year bond yields, were declined in response to both an unanticipated conventional easing and unconventional monetary actions of the Fed. While conventional monetary easing steepens the yield curve in advanced economies through a larger decline in the short-end of the yield curve, unconventional monetary actions narrow the yield spread of nominal foreign interest rates down. Moreover, \citet{hanson2015monetary} documented that the changes in monetary policy affect the 10-year forward real rates, utilizing movements during the FOMC announcement days. They offer a ``reaching for yield'' mechanism such that the yield-oriented investors substitute for longer-term bonds as short-term yields decline if the yield curve is upward-sloping. In turn, increasing demand for longer maturity bonds leads to increasing prices and declining yields. That is, this explanation relies on a ``term premium'' effect. \citep{hanson2015monetary}\\

% Add a few studies similar to Gilchrist and Hanson


Starting from \citet{romer2004new} and \citet{gurkaynak2005}, there is a growing literature on identifying high-frequency monetary policy shocks. \citet{nakamura2018high} provide a comprehensive analysis of the ``Fed information effect'' and monetary non-neutrality by examining high-frequency interest rate changes around FOMC meetings, i.e., monetary policy announcements influence not only financial variables but also adjust the beliefs and expectations of private sector participants about economic trajectory. By incorporating the adjustment of private sector expectations, the authors reveal that a considerable portion of the observed responses in real interest rates can be attributed to changes in perceptions of the natural rate of interest, emphasizing the dual role of monetary policy in shaping expectations. Yet, later on, \citet{bauer2023alternative} challenge the prevailing "Fed information effect" hypothesis, which posits that monetary policy announcements convey new information about economic conditions to the market. By analyzing high-frequency financial data around FOMC meetings and incorporating public economic news in their regressions, they propose the "Fed response to news" hypothesis. Their findings suggest that both the Fed and market participants react similarly to public economic information rather than the Fed possessing unique, market-moving insights. \\

% hillenbrand'ı kopyala yapıştır yaptım
\citet{hillenbrand2022} states that a narrow window around Fed meetings captures the secular decline in U.S. Treasury yields since 1980. Yield movements outside this window are transitory and wash out over time. This is surprising because the forces behind the secular decline are thought to be independent of monetary policy. However, Fed announcements might provide guidance about the long-run path of interest rates. In direct support of such“Long-run Fed Guidance”. \\

Heretofore, I discussed the literature on the monetary policy and declining real interest rates, depicted in bond yields. Nevertheless, another line of research that is indispensable for this study, is on the ``Global Financial Cycle'', which is elaborated in \citet{miranda2015world}, \citet{miranda2020us} and \citet{rey2021}. \citet{miranda2020us} demonstrated that a single global factor explains around one-fifth of a common variation of the risky asset prices around the world. Given that the U.S. dollar is the dominant currency of global banking, one instance of this is that almost \%80 of the syndicated loans that have an average amount greater than \$5 million are denominated in the U.S. dollar, the monetary policy decisions by the Fed have a direct impact over the Global Financial Cycle. The potential explanations for this phenomenon are the deleveraging of the financial intermediaries around the globe, and relatedly, a decline in global credit and gross capital flows, and a significant rise in aggregate risk aversion. \citep{miranda2020us} In their work on surrender options in life insurance and market interest rates, \citet{kubitza2023life} estimates two-stage least-squares regressing German government bond rates on the U.S. monetary policy shocks, claiming a transmission through the international bond market channel.
